\documentclass{article}
\usepackage[utf8]{inputenc}
\usepackage{hyperref}
\usepackage{listings}
\usepackage{xcolor} % For custom colors
\usepackage{sectsty} % For customizing section fonts
% \subsectionfont{\raggedright} % Allow line breaks and ragged right alignment             
\usepackage{geometry}
% \geometry{margin=1in}
                
\title{Functions of the TexTOM module}
\author{Moritz Frewein}
\begin{document}
\maketitle
\label{toc}
\tableofcontents

% Define a custom style for docstrings
\lstdefinelanguage{docstring}{
    basicstyle=\ttfamily\small, % Monospaced font
    backgroundcolor=\color[HTML]{F5F5F5}, % Light gray background
    frame=single, % Border around the docstring
    rulecolor=\color[HTML]{D6D6D6}, % Border color
    keywordstyle=\color{blue}, % Optional: color for keywords
    breaklines=true, % Wrap long lines
}
                
\subsectionfont{\large\ttfamily\raggedright}

\section{Introduction}

Microsimulation models are essential tools for analyzing the distributional impacts of tax and transfer policies. These models require microdata that accurately represent both the demographic composition of a population and their economic circumstances, particularly their tax situations. However, available data sources typically excel in one dimension while falling short in another.

The Current Population Survey (CPS), conducted by the U.S. Census Bureau, provides rich demographic detail and household relationships but suffers from underreporting of income and lacks tax information. Conversely, the Internal Revenue Service's Public Use File (PUF) offers precise tax data but contains limited demographic information and obscures household structure. This tradeoff between demographic detail and tax precision poses a significant challenge for policy analysis.

This paper presents a novel approach to combining these complementary data sources. We develop a methodology that preserves the demographic richness of the CPS while incorporating the tax precision of the PUF, creating an enhanced dataset that serves as the foundation for PolicyEngine's microsimulation capabilities. Our approach differs from previous efforts in three key ways:

First, we employ quantile regression forests to transfer distributions rather than point estimates between datasets, preserving the complex relationships between variables. Second, we maintain household structure throughout the enhancement process, ensuring that family relationships crucial for benefit calculations remain intact. Third, we implement a sophisticated reweighting procedure that simultaneously matches dozens of demographic and economic targets while avoiding overfitting through a dropout-enhanced gradient descent approach.

The resulting dataset demonstrates superior performance in both tax and transfer policy simulation. When compared to administrative totals, our enhanced dataset reduces discrepancies in key tax components by an average of 40\% relative to the baseline CPS, while maintaining or improving the accuracy of demographic and program participation variables.

The remainder of this paper is organized as follows: Section 2 reviews related work in survey enhancement and microsimulation data construction. Section 3 describes our data sources and their characteristics. Section 4 presents our methodology in detail. Section 5 validates our results against external benchmarks. Section 6 discusses implications and limitations, and Section 7 concludes.

Our contributions include:
\begin{itemize}
    \item A novel methodology for combining survey and administrative data while preserving distributional relationships
    \item An open-source implementation that can be adapted for other jurisdictions and policy models
    \item A validation framework comparing enhanced estimates against multiple external benchmarks
    \item A new, publicly available microdata file suitable for US tax and benefit policy analysis
\end{itemize}
\section{Functions}\label{sec:functions}

\subsection*{\texttt{set\_path(path)}}
\addcontentsline{toc}{subsection}{set\_path}

\begin{lstlisting}[language=docstring]
Set the path where integrated data and analysis is stored

Parameters
----------
path : str
    full path to the directory, must contain a folder '/data_integrated'
\end{lstlisting}

\begin{flushright}

\hyperref[toc]{ToC}

\end{flushright}

\input{functions/set_path}

\vspace{5mm}

\hrule

\subsection*{\texttt{check\_state()}}
\addcontentsline{toc}{subsection}{check\_state}

\begin{lstlisting}[language=docstring]
Prints in terminal which parts of the reconstruction are ready
    
\end{lstlisting}

\begin{flushright}

\hyperref[toc]{ToC}

\end{flushright}

\input{functions/check_state}

\vspace{5mm}

\hrule

\subsection*{\texttt{integrate()}}
\addcontentsline{toc}{subsection}{integrate}

\begin{lstlisting}[language=docstring]
Starts integrating raw data via pyFAI
    
\end{lstlisting}

\begin{flushright}

\hyperref[toc]{ToC}

\end{flushright}

\input{functions/integrate}

\vspace{5mm}

\hrule

\subsection*{\texttt{align\_data(pattern='.h5', sub\_data='data\_integrated', q\_index\_range=(0, 5), q\_range=False, crop\_image=False, mode='optical\_flow', redo\_import=False, flip\_fov=False, regroup\_max=16, align\_horizontal=True, align\_vertical=True, pre\_rec\_it=5, pre\_max\_it=5, last\_rec\_it=40, last\_max\_it=5)}}
\addcontentsline{toc}{subsection}{align\_data}

\begin{lstlisting}[language=docstring]
Align data using the Mumott optical flow alignment

Requires that data has been integrated and that sample_dir contains
a subfolder with data

Parameters
----------
pattern : str, optional
    substring contained in all files you want to use, by default '.h5'
sub_data : str, optional
    subfolder containing the data, by default 'data_integrated'
q_index_range : tuple, optional
    determines which q-values are used for alignment (sums over them), by default (0,5)
q_range : tuple, optional
    give the q-range in nm instead of indices e.g. (15.8,18.1), by default False
crop_image : bool or tuple of int, optional
    give the range you want to use in x and y, e.g. ((0,-1),(10,-10)), by default False
mode : str, optional
    choose alignment mode, 'optical_flow' or 'phase_matching', by default 'optical_flow'
redo_import : bool, optional
    set True if you want to recalculate data_mumott.h5, by default False
flip_fov : bool, optional
    only to be used if the fov is in the wrong order in the integrated
    data files, by default False
regroup_max : int, optional
    maximum size of groups when downsampling for faster processing, by default 16
align_horizontal : bool, optional
    align your data horizontally, by default True
align_vertical : bool, optional
    align your data vertically, by default True
pre_rec_it : int, optional
    reconstruciton iterations for downsampled data, by default 5
pre_max_it : int, optional
    alignment iterations for downsampled data, by default 5
last_rec_it : int, optional
    reconstruciton iterations for full data, by default 40
last_max_it : int, optional
    alignment iterations for full data, by default 5
\end{lstlisting}

\begin{flushright}

\hyperref[toc]{ToC}

\end{flushright}

\input{functions/align_data}

\vspace{5mm}

\hrule

\subsection*{\texttt{check\_alignment\_consistency()}}
\addcontentsline{toc}{subsection}{check\_alignment\_consistency}

\begin{lstlisting}[language=docstring]
Plots the squared residuals between data and the projected tomograms

    
\end{lstlisting}

\begin{flushright}

\hyperref[toc]{ToC}

\end{flushright}

\input{functions/check_alignment_consistency}

\vspace{5mm}

\hrule

\subsection*{\texttt{check\_alignment\_projection(g=0)}}
\addcontentsline{toc}{subsection}{check\_alignment\_projection}

\begin{lstlisting}[language=docstring]
Plots the data and the projected tomogram of projection g

Parameters
----------
g : int, optional
    projection running index, by default 0
\end{lstlisting}

\begin{flushright}

\hyperref[toc]{ToC}

\end{flushright}

\input{functions/check_alignment_projection}

\vspace{5mm}

\hrule

\subsection*{\texttt{make\_model()}}
\addcontentsline{toc}{subsection}{make\_model}

\begin{lstlisting}[language=docstring]
Calculates the TexTOM model for reconstructions

Is automatically performed by the functions that require it
\end{lstlisting}

\begin{flushright}

\hyperref[toc]{ToC}

\end{flushright}

\input{functions/make_model}

\vspace{5mm}

\hrule

\subsection*{\texttt{preprocess\_data(pattern='.h5', flip\_fov=False, baselines=True, use\_ion=True)}}
\addcontentsline{toc}{subsection}{preprocess\_data}

\begin{lstlisting}[language=docstring]
Loads integrated data and pre-processes them for TexTOM

Parameters
----------
pattern : str, optional
    substring contained in all files you want to use, by default '.h5'
flip_fov : bool, optional
    only to be used if the fov is in the wrong order in the integrated
    data files, by default False
baselines : bool, optional
    choose if you want to draw polynomial baselines, by default True
use_ion : bool, optional
    choose if you want to normalize data by the field 'ion' in the 
    data files, by default True
\end{lstlisting}

\begin{flushright}

\hyperref[toc]{ToC}

\end{flushright}

\input{functions/preprocess_data}

\vspace{5mm}

\hrule

\subsection*{\texttt{make\_fit(redo=True)}}
\addcontentsline{toc}{subsection}{make\_fit}

\begin{lstlisting}[language=docstring]
Initializes a TexTOM fit object for reconstructions

Is automatically performed by the functions that require it

Parameters
----------
redo : bool, optional
    set True for recalculating, by default True
\end{lstlisting}

\begin{flushright}

\hyperref[toc]{ToC}

\end{flushright}

\input{functions/make_fit}

\vspace{5mm}

\hrule

\subsection*{\texttt{optimize(order=0, mode=0, proj='full', redo\_fit=False, tol=0.001, minstep=1e-09, itermax=3000, alg='quadratic', save\_h5=True)}}
\addcontentsline{toc}{subsection}{optimize}

\begin{lstlisting}[language=docstring]
Performs a single TexTOM parameter optimization

Parameters
----------
order : int, optional
    maximum sHSH order to be used, by default 0
mode : int, optional
    set 0 for only optimizing order 0, 1 for highest order, 2 for all,
    by default 0
proj : str, optional
    choose projections to be optimized: 'full', 'half', 'third', 'notilt', 
    by default 'full'
redo_fit : bool, optional
    recalculate the fit object, by default False
tol : float, optional
    tolerance for precision break criterion, by default 1e-3
minstep : float, optional
    minimum stepsize in line search, by default 1e-9
itermax : int, optional
    maximum number of iterations, by default 3000
alg : str, optional
    choose algorithm between 'backtracking', 'simple', 'quadratic', 
    by default 'quadratic'
save_h5 : bool, optional
    choose if you want to save the result to the directory analysis/fits, 
    by default True    
\end{lstlisting}

\begin{flushright}

\hyperref[toc]{ToC}

\end{flushright}

\input{functions/optimize}

\vspace{5mm}

\hrule

\subsection*{\texttt{optimize\_auto(max\_order=8, start\_order=None, tol\_0=1e-07, tol\_1=0.001, tol\_2=0.0001, minstep\_0=1e-09, minstep\_1=1e-09, minstep\_2=1e-09, projections='full', alg='quadratic', adj\_scal=False, redo\_fit=False)}}
\addcontentsline{toc}{subsection}{optimize\_auto}

\begin{lstlisting}[language=docstring]
Automated TexTOM reconstruction workflow

Parameters
----------
max_order : int, optional
    maximum HSH order to be used, by default 8    
start_order : int or None, optional
    lowest order to be fitted, if None continues where you are standing, 
    by default None
redo_fit : bool, optional
    recalculate the fit object, by default False
proj : str, optional
    choose projections to be optimized: 'full', 'half', 'third', 'notilt', by default 'full'
alg : str, optional
    choose algorithm between 'backtracking', 'simple', 'quadratic', 
    by default 'quadratic'
\end{lstlisting}

\begin{flushright}

\hyperref[toc]{ToC}

\end{flushright}

\input{functions/optimize_auto}

\vspace{5mm}

\hrule

\subsection*{\texttt{list\_opt()}}
\addcontentsline{toc}{subsection}{list\_opt}

\begin{lstlisting}[language=docstring]
Shows all stored optimizations
    
\end{lstlisting}

\begin{flushright}

\hyperref[toc]{ToC}

\end{flushright}

\input{functions/list_opt}

\vspace{5mm}

\hrule

\subsection*{\texttt{load\_opt(h5path='last')}}
\addcontentsline{toc}{subsection}{load\_opt}

\begin{lstlisting}[language=docstring]
Loads a previous Textom optimization into memory
seful: load_opt(results['optimization'])

Parameters
----------
h5path : str, optional
    filepath, just filename or full path
    if 'last', uses the youngest file is used in analysis/fits/, 
    by default 'last'
\end{lstlisting}

\begin{flushright}

\hyperref[toc]{ToC}

\end{flushright}

\input{functions/load_opt}

\vspace{5mm}

\hrule

\subsection*{\texttt{check\_lossfunction()}}
\addcontentsline{toc}{subsection}{check\_lossfunction}

\begin{lstlisting}[language=docstring]
No docstring available.
\end{lstlisting}

\begin{flushright}

\hyperref[toc]{ToC}

\end{flushright}

\input{functions/check_lossfunction}

\vspace{5mm}

\hrule

\subsection*{\texttt{check\_fit\_average()}}
\addcontentsline{toc}{subsection}{check\_fit\_average}

\begin{lstlisting}[language=docstring]
Plots the reconstructed average intensity for each projection with data

Parameters
----------
\end{lstlisting}

\begin{flushright}

\hyperref[toc]{ToC}

\end{flushright}

\input{functions/check_fit_average}

\vspace{5mm}

\hrule

\subsection*{\texttt{check\_fit\_random(N=10, mode='line')}}
\addcontentsline{toc}{subsection}{check\_fit\_random}

\begin{lstlisting}[language=docstring]
Generates TexTOM reconstructions and plots them with data for random points

Parameters
----------
N : int, optional
    Number of images created, by default 10    
mode : str, optional
    plotting mode, 'line' or 'color', by default line
\end{lstlisting}

\begin{flushright}

\hyperref[toc]{ToC}

\end{flushright}

\input{functions/check_fit_random}

\vspace{5mm}

\hrule

\subsection*{\texttt{check\_residuals()}}
\addcontentsline{toc}{subsection}{check\_residuals}

\begin{lstlisting}[language=docstring]
Plots the squared residuals summed over each projection
    
\end{lstlisting}

\begin{flushright}

\hyperref[toc]{ToC}

\end{flushright}

\input{functions/check_residuals}

\vspace{5mm}

\hrule

\subsection*{\texttt{check\_projections\_average(G=None)}}
\addcontentsline{toc}{subsection}{check\_projections\_average}

\begin{lstlisting}[language=docstring]
Plots the reconstructed average intensity for chosen projections with data

Parameters
----------
G : int or ndarray or None, optional
    projection indices, if None takes 10 equidistant ones, by default None
\end{lstlisting}

\begin{flushright}

\hyperref[toc]{ToC}

\end{flushright}

\input{functions/check_projections_average}

\vspace{5mm}

\hrule

\subsection*{\texttt{check\_projections\_residuals(G=None)}}
\addcontentsline{toc}{subsection}{check\_projections\_residuals}

\begin{lstlisting}[language=docstring]
Plots the residuals per pix3l for chosen projections with data

Parameters
----------
G : int or ndarray or None, optional
    projection indices, if None takes 10 equidistant ones, by default None
\end{lstlisting}

\begin{flushright}

\hyperref[toc]{ToC}

\end{flushright}

\input{functions/check_projections_residuals}

\vspace{5mm}

\hrule

\subsection*{\texttt{check\_projections\_orientations(G=None)}}
\addcontentsline{toc}{subsection}{check\_projections\_orientations}

\begin{lstlisting}[language=docstring]
Plots the reconstructed average orientations for chosen projections with data

Parameters
----------
G : int or ndarray or None, optional
    projection indices, if None takes 10 equidistant ones, by default None
\end{lstlisting}

\begin{flushright}

\hyperref[toc]{ToC}

\end{flushright}

\input{functions/check_projections_orientations}

\vspace{5mm}

\hrule

\subsection*{\texttt{calculate\_orientation\_statistics()}}
\addcontentsline{toc}{subsection}{calculate\_orientation\_statistics}

\begin{lstlisting}[language=docstring]
Calculates prefered orientations and stds and saves them to results dict

    
\end{lstlisting}

\begin{flushright}

\hyperref[toc]{ToC}

\end{flushright}

\input{functions/calculate_orientation_statistics}

\vspace{5mm}

\hrule

\subsection*{\texttt{calculate\_segments(thresh=10, min\_segment\_size=30, max\_segments\_number=31)}}
\addcontentsline{toc}{subsection}{calculate\_segments}

\begin{lstlisting}[language=docstring]
Segments the sample based on misorientation borders

Parameters
----------
thresh : float, optional
    misorientation angle threshold inside segment in degree, by default 10
min_segment_size : int, optional
    minimum number of voxels in segment, by default 30
max_segments_number : int, optional
    maximum number of segments (ordered by size), by default 32
\end{lstlisting}

\begin{flushright}

\hyperref[toc]{ToC}

\end{flushright}

\input{functions/calculate_segments}

\vspace{5mm}

\hrule

\subsection*{\texttt{show\_volume(data='scaling', plane='z', colormap='inferno', cut=1, save=False, show=True)}}
\addcontentsline{toc}{subsection}{show\_volume}

\begin{lstlisting}[language=docstring]
Visualizes the whole sample by slices, colored by a value of your choice

Parameters
----------
data : str or list, optional
    name of one entry in the results dict or list of entries, 
    by default 'scaling'
plane : str, optional
    sliceplane 'x'/'y'/'z', by default 'z'
colormap : str, optional
    identifier of matplotlib colormap, default 'inferno'
    https://matplotlib.org/stable/users/explain/colors/colormaps.html
cut : int, optional
    cut colorscale at upper and lower percentile, by default 0.1
\end{lstlisting}

\begin{flushright}

\hyperref[toc]{ToC}

\end{flushright}

\input{functions/show_volume}

\vspace{5mm}

\hrule

\subsection*{\texttt{show\_slice\_ipf(h, plane='z')}}
\addcontentsline{toc}{subsection}{show\_slice\_ipf}

\begin{lstlisting}[language=docstring]
Plots an inverse pole figure of a sample slice

Parameters
----------
h : int
    height of the slice
plane : str, optional
    slice direction: x/y/z, by default 'z'
\end{lstlisting}

\begin{flushright}

\hyperref[toc]{ToC}

\end{flushright}

\input{functions/show_slice_ipf}

\vspace{5mm}

\hrule

\subsection*{\texttt{show\_volume\_ipf(plane='z', save=False, show=True)}}
\addcontentsline{toc}{subsection}{show\_volume\_ipf}

\begin{lstlisting}[language=docstring]
Plots inverse pole figures as a tomogram with a slider to scroll through the sample

Parameters
----------
plane : str, optional
    slice direction: x/y/z, by default 'z'
save : bool, optional
    if True, saves movie as .gif to results/images, by default False
show : bool, optional
    if True, opens matplotlib window, by default True
\end{lstlisting}

\begin{flushright}

\hyperref[toc]{ToC}

\end{flushright}

\input{functions/show_volume_ipf}

\vspace{5mm}

\hrule

\subsection*{\texttt{show\_histogram(x, nbins=50, cut=0.1, segments=None, save=False)}}
\addcontentsline{toc}{subsection}{show\_histogram}

\begin{lstlisting}[language=docstring]
plots a histogram of a result parameter

Parameters
----------
x : str,
    name of a scalar from results
bins : int, optional
    number of bins, by default 50
cut : int, optional
    cut upper and lower percentile, by default 0.1
segments : list of int, optional
    list of segments or None for all data, by default None
\end{lstlisting}

\begin{flushright}

\hyperref[toc]{ToC}

\end{flushright}

\input{functions/show_histogram}

\vspace{5mm}

\hrule

\subsection*{\texttt{show\_correlations(x, y, nbins=50, cut=(0.1, 0.1), segments=None, save=False)}}
\addcontentsline{toc}{subsection}{show\_correlations}

\begin{lstlisting}[language=docstring]
plots a 2D histogram between 2 result parameters

Parameters
----------
x : str,
    name of a scalar from results
y : str,
    name of a scalar from results
bins : int, optional
    number of bins, by default 50
cut : tuple, optional
    cut upper and lower percentile of both parameters, by default (0.1,0.1)
segments : list, optional
    list of segments or None for all data, by default None
\end{lstlisting}

\begin{flushright}

\hyperref[toc]{ToC}

\end{flushright}

\input{functions/show_correlations}

\vspace{5mm}

\hrule

\subsection*{\texttt{show\_voxel\_odf(x, y, z, num\_samples=1000)}}
\addcontentsline{toc}{subsection}{show\_voxel\_odf}

\begin{lstlisting}[language=docstring]
Show a 3D plot of the ODF in the chosen voxel

Parameters
----------
x : int
    voxel x-coordinate
y : int
    voxel y-coordinate
z : int
    voxel z-coordinate
\end{lstlisting}

\begin{flushright}

\hyperref[toc]{ToC}

\end{flushright}

\input{functions/show_voxel_odf}

\vspace{5mm}

\hrule

\subsection*{\texttt{show\_voxel\_polefigure(x, y, z, hkl=(1, 0, 0), mode='density', alpha=0.1, num\_samples=10000.0)}}
\addcontentsline{toc}{subsection}{show\_voxel\_polefigure}

\begin{lstlisting}[language=docstring]
Show a polefigure plot for the chosen voxel and hkl

Parameters
----------
x : int
    voxel x-coordinate
y : int
    voxel y-coordinate
z : int
    voxel z-coordinate
hkl : tuple, optional
    Miller indices, by default (1,0,0)
mode : str, optional
    plotting style 'scatter' or 'density', by default 'density'
alpha : float, optional
    opacity of points, only for scatter, by default 0.1
num_samples : int/float, optional
    number of samples for plot generation, by default 1e4
\end{lstlisting}

\begin{flushright}

\hyperref[toc]{ToC}

\end{flushright}

\input{functions/show_voxel_polefigure}

\vspace{5mm}

\hrule

\subsection*{\texttt{reconstruct\_1d\_full(redo\_import=False, only\_mumottize=False, batch\_size=10)}}
\addcontentsline{toc}{subsection}{reconstruct\_1d\_full}

\begin{lstlisting}[language=docstring]
Reconstructs standard tomographic data such as azimutally averaged
diffraction data. Uses the same alignment as textom

Parameters
----------
redo_import : bool, optional
    _description_, by default False
only_mumottize : bool, optional
    only preprocesses a file analysis/rec1d/data_rec1d.h5, by default False
batch_size : int, optional
    number of q-values to load at the same time. Needs to be an integer fraction
    of the total number of q-values, else it will crash at the last batch. Higher
    numbers will decrease i/o time, but require more memory, by default 10
\end{lstlisting}

\begin{flushright}

\hyperref[toc]{ToC}

\end{flushright}

\input{functions/reconstruct_1d_full}

\vspace{5mm}

\hrule

\subsection*{\texttt{save\_results()}}
\addcontentsline{toc}{subsection}{save\_results}

\begin{lstlisting}[language=docstring]
Saves the results dictionary to a h5 file

    
\end{lstlisting}

\begin{flushright}

\hyperref[toc]{ToC}

\end{flushright}

\input{functions/save_results}

\vspace{5mm}

\hrule

\subsection*{\texttt{link\_xdmf(paths)}}
\addcontentsline{toc}{subsection}{link\_xdmf}

\begin{lstlisting}[language=docstring]
No docstring available.
\end{lstlisting}

\begin{flushright}

\hyperref[toc]{ToC}

\end{flushright}

\input{functions/link_xdmf}

\vspace{5mm}

\hrule

\subsection*{\texttt{list\_results()}}
\addcontentsline{toc}{subsection}{list\_results}

\begin{lstlisting}[language=docstring]
Shows all results .h5 files in results directory
    
\end{lstlisting}

\begin{flushright}

\hyperref[toc]{ToC}

\end{flushright}

\input{functions/list_results}

\vspace{5mm}

\hrule

\subsection*{\texttt{load\_results(h5path='last', make\_bg\_nan=False)}}
\addcontentsline{toc}{subsection}{load\_results}

\begin{lstlisting}[language=docstring]
Loads the results from a h5 file do the results dictionary

    
\end{lstlisting}

\begin{flushright}

\hyperref[toc]{ToC}

\end{flushright}

\input{functions/load_results}

\vspace{5mm}

\hrule

\subsection*{\texttt{list\_results\_loaded()}}
\addcontentsline{toc}{subsection}{list\_results\_loaded}

\begin{lstlisting}[language=docstring]
Shows all results currently in memory
    
\end{lstlisting}

\begin{flushright}

\hyperref[toc]{ToC}

\end{flushright}

\input{functions/list_results_loaded}

\vspace{5mm}

\hrule

\subsection*{\texttt{save\_images(x, ext='raw')}}
\addcontentsline{toc}{subsection}{save\_images}

\begin{lstlisting}[language=docstring]
Export results as .raw or .tiff files for dragonfly

Parameters
----------
x : str,
    name of a scalar from results, e.g. 'scaling'
\end{lstlisting}

\begin{flushright}

\hyperref[toc]{ToC}

\end{flushright}

\input{functions/save_images}

\vspace{5mm}

\hrule

\subsection*{\texttt{sort\_integrated\_files(source='data\_integrated', target='data\_integrated\_1d', check\_end=-6, pattern='.h5')}}
\addcontentsline{toc}{subsection}{sort\_integrated\_files}

\begin{lstlisting}[language=docstring]
Reads all filenames in source folder and checks if the same exist in target
Moves the non-overlapping to an 'excluded'-subfolder 

Parameters
----------
source : str, optional
    directory with sorted files, by default 'data_integrated'
target : str, optional
    directory with files to sort, by default 'data_integrated_1d'
check_end : int, optional
    index of the last character of the filename to check, by default -6
pattern : str, optional
    only filenames that contain this are searched, by default '.h5'
\end{lstlisting}

\begin{flushright}

\hyperref[toc]{ToC}

\end{flushright}

\input{functions/sort_integrated_files}

\vspace{5mm}

\hrule

\subsection*{\texttt{help(method=None, module=None, filter='')}}
\addcontentsline{toc}{subsection}{help}

\begin{lstlisting}[language=docstring]
Prints information about functions in this library

Parameters
----------
method : str or None, optional
    get more information about a function or None for overview over all functions, by default None
module : str or None, optional
    choose python module or None for the base TexTOM library, by default None
filter : str, optional
    filter the displayed functions by a substring, by default ''
\end{lstlisting}

\begin{flushright}

\hyperref[toc]{ToC}

\end{flushright}

\input{functions/help}

\vspace{5mm}

\hrule

\end{document}\end{document}