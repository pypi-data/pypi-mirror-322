\section{Conclusion}

This paper presents a novel approach to constructing enhanced microdata for tax-benefit microsimulation by combining survey and administrative data sources. Our methodology leverages machine learning techniques – specifically quantile regression forests and gradient descent optimization – to preserve the strengths of each source while mitigating their weaknesses. The resulting dataset outperforms both the Current Population Survey and IRS Public Use File across a majority of validation targets, with particularly strong improvements in areas crucial for policy analysis such as income distributions and program participation rates.

The enhanced dataset addresses a key challenge in tax-benefit microsimulation: the need for both detailed demographic information and accurate tax/income data. By maintaining the CPS's rich household structure while incorporating the PUF's tax precision, our approach enables more reliable analysis of policies that depend on both demographic characteristics and economic circumstances. The systematic validation against hundreds of administrative targets provides confidence in the dataset's reliability while helping users understand its limitations.

Our open-source implementation and automatically updated validation metrics establish a new standard for transparency in microsimulation data enhancement. This enables other researchers to build upon our work, adapt the methodology to other jurisdictions, or extend it to incorporate additional data sources. Future work could expand the approach to finer geographic levels, integrate data from additional surveys, or apply similar techniques to other domains requiring the combination of survey and administrative data.

The enhanced CPS represents a significant advance in the quality of openly available microdata for tax-benefit analysis. By reducing error rates across a broad range of metrics while preserving essential relationships in the data, it provides a more reliable foundation for understanding the impacts of complex policy reforms on American households.