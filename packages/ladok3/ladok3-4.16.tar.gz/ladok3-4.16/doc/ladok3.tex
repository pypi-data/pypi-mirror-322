\documentclass[a4paper,oneside]{book}
\newenvironment{abstract}{}{}
\usepackage{abstract}
\usepackage{noweb}
% Needed to relax penalty for breaking code chunks across pages, otherwise 
% there might be a lot of space following a code chunk.
\def\nwendcode{\endtrivlist \endgroup}
\let\nwdocspar=\smallbreak

\usepackage[hyphens]{url}
\usepackage{hyperref}
\usepackage{authblk}

\usepackage[utf8]{inputenc}
\usepackage[T1]{fontenc}
\usepackage[swedish,british]{babel}
\usepackage{booktabs}

\usepackage[natbib,style=alphabetic,maxbibnames=99]{biblatex}
\addbibresource{ladok_lib.bib}

\usepackage[inline]{enumitem}
\setlist[enumerate]{label=(\arabic*)}

\usepackage[all]{foreign}
\renewcommand{\foreignfullfont}{}
\renewcommand{\foreignabbrfont}{}

%\usepackage{newclude}
\usepackage{import}

\usepackage[strict]{csquotes}
\usepackage[single]{acro}

\usepackage{subcaption}

\usepackage[noend]{algpseudocode}
\usepackage{xparse}

\let\email\texttt

\usepackage[outputdir=ltxobj]{minted}
\setminted{autogobble,linenos}

\usepackage{pythontex}
\setpythontexoutputdir{.}
\setpythontexworkingdir{..}

\usepackage{fancyvrb}

\usepackage{amsmath}
\usepackage{amssymb}
\usepackage{mathtools}
\usepackage{amsthm}
\usepackage{thmtools}
\usepackage[unq]{unique}
\DeclareMathOperator{\powerset}{\mathcal{P}}

\usepackage[binary-units]{siunitx}

\usepackage{pgf}

\usepackage[capitalize]{cleveref}
\crefalias{question}{item}
\crefname{question}{question}{questions}
\Crefname{question}{Question}{Questions}
\creflabelformat{question}{#2\textup{#1}#3}



\title{%
  A LADOK3 Python API
}
\author{%
  Alexander Baltatzis,
  Daniel Bosk,
  Gerald Q.\ \enquote{Chip} Maguire Jr.
}
\affil{%
  KTH EECS\\
  \texttt{\{alba,dbosk,maguire\}@kth.se}
}

\begin{document}
\frontmatter
\maketitle

\vspace*{\fill}
\VerbatimInput{../LICENSE}
\clearpage

\begin{abstract}
  We provide a Python wrapper for the LADOK3 REST API\@.
We provide a useful object-oriented framework and direct API calls that return 
the unprocessed JSON data from LADOK.

\end{abstract}
\clearpage

\tableofcontents
\clearpage

\mainmatter
\chapter{Introduction}

LADOK (abbreviation of \foreignlanguage{swedish}{Lokalt Adb–baserat 
DOKumentationssystem}, in Swedish) is the national documentation system for 
higher education in Sweden.
This is the documented source code of \texttt{ladok3}, a LADOK3 API wrapper for 
Python.

The \texttt{ladok3} library provides a non-GUI application that, similar to 
access via a web browser, only provides the user with access to the LADOK data 
and functionality that they would actually have based on their specific user 
permissions in LADOK.
It can be seem as a very streamlined web browser just for LADOK's web 
interface.
While the library exploits caching to reduce the load on the LADOK server, this 
represents a subset of the information that would otherwise be obtained via
LADOK's web GUI export functions.

You can install the \texttt{ladok3} package by running
\begin{minted}{bash}
pip3 install ladok3
\end{minted}
in the terminal.
You can find a quick reference by running
\begin{minted}[firstnumber=last]{bash}
pydoc ladok3
\end{minted}

We provide the main class \texttt{LadokSession} (\cref{LadokSession}).
The \texttt{LadokSession} class acts like \enquote{factories} and will return 
objects representing various LADOK data.
These data objects' classes inherit the \texttt{LadokData} (\cref{LadokData}) 
and \texttt{LadokRemoteData} (\cref{LadokRemoteData}) classes.
Data of the type \texttt{LadokData} is not expected to change, unlike 
\texttt{LadokRemoteData}, which is.
Objects of type \texttt{LadokRemoteData} know how to update themselves, \ie fetch 
and refresh their data from LADOK.
When relevant they can also write data to LADOK, \ie update entries such as 
results.

One design criteria is to improve performance.
We do this by caching all factory methods of any \texttt{LadokSession}.
The \texttt{LadokSession} itself is also designed to be cacheable: if the session to 
LADOK expires, it will automatically reauthenticate to establish a new session.



\part{The library}

\documentclass[a4paper,oneside]{book}
\newenvironment{abstract}{}{}
\usepackage{abstract}
\usepackage{noweb}
% Needed to relax penalty for breaking code chunks across pages, otherwise 
% there might be a lot of space following a code chunk.
\def\nwendcode{\endtrivlist \endgroup}
\let\nwdocspar=\smallbreak

\usepackage[hyphens]{url}
\usepackage{hyperref}
\usepackage{authblk}

\usepackage[utf8]{inputenc}
\usepackage[T1]{fontenc}
\usepackage[swedish,british]{babel}
\usepackage{booktabs}

\usepackage[natbib,style=alphabetic,maxbibnames=99]{biblatex}
\addbibresource{ladok_lib.bib}

\usepackage[inline]{enumitem}
\setlist[enumerate]{label=(\arabic*)}

\usepackage[all]{foreign}
\renewcommand{\foreignfullfont}{}
\renewcommand{\foreignabbrfont}{}

%\usepackage{newclude}
\usepackage{import}

\usepackage[strict]{csquotes}
\usepackage[single]{acro}

\usepackage{subcaption}

\usepackage[noend]{algpseudocode}
\usepackage{xparse}

\let\email\texttt

\usepackage[outputdir=ltxobj]{minted}
\setminted{autogobble,linenos}

\usepackage{pythontex}
\setpythontexoutputdir{.}
\setpythontexworkingdir{..}

\usepackage{fancyvrb}

\usepackage{amsmath}
\usepackage{amssymb}
\usepackage{mathtools}
\usepackage{amsthm}
\usepackage{thmtools}
\usepackage[unq]{unique}
\DeclareMathOperator{\powerset}{\mathcal{P}}

\usepackage[binary-units]{siunitx}

\usepackage{pgf}

\usepackage[capitalize]{cleveref}
\crefalias{question}{item}
\crefname{question}{question}{questions}
\Crefname{question}{Question}{Questions}
\creflabelformat{question}{#2\textup{#1}#3}



\title{%
  A LADOK3 Python API
}
\author{%
  Alexander Baltatzis,
  Daniel Bosk,
  Gerald Q.\ \enquote{Chip} Maguire Jr.
}
\affil{%
  KTH EECS\\
  \texttt{\{alba,dbosk,maguire\}@kth.se}
}

\begin{document}
\frontmatter
\maketitle

\vspace*{\fill}
\VerbatimInput{../LICENSE}
\clearpage

\begin{abstract}
  We provide a Python wrapper for the LADOK3 REST API\@.
We provide a useful object-oriented framework and direct API calls that return 
the unprocessed JSON data from LADOK.

\end{abstract}
\clearpage

\tableofcontents
\clearpage

\mainmatter
\chapter{Introduction}

LADOK (abbreviation of \foreignlanguage{swedish}{Lokalt Adb–baserat 
DOKumentationssystem}, in Swedish) is the national documentation system for 
higher education in Sweden.
This is the documented source code of \texttt{ladok3}, a LADOK3 API wrapper for 
Python.

The \texttt{ladok3} library provides a non-GUI application that, similar to 
access via a web browser, only provides the user with access to the LADOK data 
and functionality that they would actually have based on their specific user 
permissions in LADOK.
It can be seem as a very streamlined web browser just for LADOK's web 
interface.
While the library exploits caching to reduce the load on the LADOK server, this 
represents a subset of the information that would otherwise be obtained via
LADOK's web GUI export functions.

You can install the \texttt{ladok3} package by running
\begin{minted}{bash}
pip3 install ladok3
\end{minted}
in the terminal.
You can find a quick reference by running
\begin{minted}[firstnumber=last]{bash}
pydoc ladok3
\end{minted}

We provide the main class \texttt{LadokSession} (\cref{LadokSession}).
The \texttt{LadokSession} class acts like \enquote{factories} and will return 
objects representing various LADOK data.
These data objects' classes inherit the \texttt{LadokData} (\cref{LadokData}) 
and \texttt{LadokRemoteData} (\cref{LadokRemoteData}) classes.
Data of the type \texttt{LadokData} is not expected to change, unlike 
\texttt{LadokRemoteData}, which is.
Objects of type \texttt{LadokRemoteData} know how to update themselves, \ie fetch 
and refresh their data from LADOK.
When relevant they can also write data to LADOK, \ie update entries such as 
results.

One design criteria is to improve performance.
We do this by caching all factory methods of any \texttt{LadokSession}.
The \texttt{LadokSession} itself is also designed to be cacheable: if the session to 
LADOK expires, it will automatically reauthenticate to establish a new session.



\part{The library}

\documentclass[a4paper,oneside]{book}
\newenvironment{abstract}{}{}
\usepackage{abstract}
\usepackage{noweb}
% Needed to relax penalty for breaking code chunks across pages, otherwise 
% there might be a lot of space following a code chunk.
\def\nwendcode{\endtrivlist \endgroup}
\let\nwdocspar=\smallbreak

\usepackage[hyphens]{url}
\usepackage{hyperref}
\usepackage{authblk}

\usepackage[utf8]{inputenc}
\usepackage[T1]{fontenc}
\usepackage[swedish,british]{babel}
\usepackage{booktabs}

\usepackage[natbib,style=alphabetic,maxbibnames=99]{biblatex}
\addbibresource{ladok_lib.bib}

\usepackage[inline]{enumitem}
\setlist[enumerate]{label=(\arabic*)}

\usepackage[all]{foreign}
\renewcommand{\foreignfullfont}{}
\renewcommand{\foreignabbrfont}{}

%\usepackage{newclude}
\usepackage{import}

\usepackage[strict]{csquotes}
\usepackage[single]{acro}

\usepackage{subcaption}

\usepackage[noend]{algpseudocode}
\usepackage{xparse}

\let\email\texttt

\usepackage[outputdir=ltxobj]{minted}
\setminted{autogobble,linenos}

\usepackage{pythontex}
\setpythontexoutputdir{.}
\setpythontexworkingdir{..}

\usepackage{fancyvrb}

\usepackage{amsmath}
\usepackage{amssymb}
\usepackage{mathtools}
\usepackage{amsthm}
\usepackage{thmtools}
\usepackage[unq]{unique}
\DeclareMathOperator{\powerset}{\mathcal{P}}

\usepackage[binary-units]{siunitx}

\usepackage{pgf}

\usepackage[capitalize]{cleveref}
\crefalias{question}{item}
\crefname{question}{question}{questions}
\Crefname{question}{Question}{Questions}
\creflabelformat{question}{#2\textup{#1}#3}



\title{%
  A LADOK3 Python API
}
\author{%
  Alexander Baltatzis,
  Daniel Bosk,
  Gerald Q.\ \enquote{Chip} Maguire Jr.
}
\affil{%
  KTH EECS\\
  \texttt{\{alba,dbosk,maguire\}@kth.se}
}

\begin{document}
\frontmatter
\maketitle

\vspace*{\fill}
\VerbatimInput{../LICENSE}
\clearpage

\begin{abstract}
  We provide a Python wrapper for the LADOK3 REST API\@.
We provide a useful object-oriented framework and direct API calls that return 
the unprocessed JSON data from LADOK.

\end{abstract}
\clearpage

\tableofcontents
\clearpage

\mainmatter
\chapter{Introduction}

LADOK (abbreviation of \foreignlanguage{swedish}{Lokalt Adb–baserat 
DOKumentationssystem}, in Swedish) is the national documentation system for 
higher education in Sweden.
This is the documented source code of \texttt{ladok3}, a LADOK3 API wrapper for 
Python.

The \texttt{ladok3} library provides a non-GUI application that, similar to 
access via a web browser, only provides the user with access to the LADOK data 
and functionality that they would actually have based on their specific user 
permissions in LADOK.
It can be seem as a very streamlined web browser just for LADOK's web 
interface.
While the library exploits caching to reduce the load on the LADOK server, this 
represents a subset of the information that would otherwise be obtained via
LADOK's web GUI export functions.

You can install the \texttt{ladok3} package by running
\begin{minted}{bash}
pip3 install ladok3
\end{minted}
in the terminal.
You can find a quick reference by running
\begin{minted}[firstnumber=last]{bash}
pydoc ladok3
\end{minted}

We provide the main class \texttt{LadokSession} (\cref{LadokSession}).
The \texttt{LadokSession} class acts like \enquote{factories} and will return 
objects representing various LADOK data.
These data objects' classes inherit the \texttt{LadokData} (\cref{LadokData}) 
and \texttt{LadokRemoteData} (\cref{LadokRemoteData}) classes.
Data of the type \texttt{LadokData} is not expected to change, unlike 
\texttt{LadokRemoteData}, which is.
Objects of type \texttt{LadokRemoteData} know how to update themselves, \ie fetch 
and refresh their data from LADOK.
When relevant they can also write data to LADOK, \ie update entries such as 
results.

One design criteria is to improve performance.
We do this by caching all factory methods of any \texttt{LadokSession}.
The \texttt{LadokSession} itself is also designed to be cacheable: if the session to 
LADOK expires, it will automatically reauthenticate to establish a new session.



\part{The library}

\documentclass[a4paper,oneside]{book}
\newenvironment{abstract}{}{}
\usepackage{abstract}
\usepackage{noweb}
% Needed to relax penalty for breaking code chunks across pages, otherwise 
% there might be a lot of space following a code chunk.
\def\nwendcode{\endtrivlist \endgroup}
\let\nwdocspar=\smallbreak

\usepackage[hyphens]{url}
\usepackage{hyperref}
\usepackage{authblk}

\input{preamble.tex}

\title{%
  A LADOK3 Python API
}
\author{%
  Alexander Baltatzis,
  Daniel Bosk,
  Gerald Q.\ \enquote{Chip} Maguire Jr.
}
\affil{%
  KTH EECS\\
  \texttt{\{alba,dbosk,maguire\}@kth.se}
}

\begin{document}
\frontmatter
\maketitle

\vspace*{\fill}
\VerbatimInput{../LICENSE}
\clearpage

\begin{abstract}
  \input{abstract.tex}
\end{abstract}
\clearpage

\tableofcontents
\clearpage

\mainmatter
\chapter{Introduction}

LADOK (abbreviation of \foreignlanguage{swedish}{Lokalt Adb–baserat 
DOKumentationssystem}, in Swedish) is the national documentation system for 
higher education in Sweden.
This is the documented source code of \texttt{ladok3}, a LADOK3 API wrapper for 
Python.

The \texttt{ladok3} library provides a non-GUI application that, similar to 
access via a web browser, only provides the user with access to the LADOK data 
and functionality that they would actually have based on their specific user 
permissions in LADOK.
It can be seem as a very streamlined web browser just for LADOK's web 
interface.
While the library exploits caching to reduce the load on the LADOK server, this 
represents a subset of the information that would otherwise be obtained via
LADOK's web GUI export functions.

You can install the \texttt{ladok3} package by running
\begin{minted}{bash}
pip3 install ladok3
\end{minted}
in the terminal.
You can find a quick reference by running
\begin{minted}[firstnumber=last]{bash}
pydoc ladok3
\end{minted}

We provide the main class \texttt{LadokSession} (\cref{LadokSession}).
The \texttt{LadokSession} class acts like \enquote{factories} and will return 
objects representing various LADOK data.
These data objects' classes inherit the \texttt{LadokData} (\cref{LadokData}) 
and \texttt{LadokRemoteData} (\cref{LadokRemoteData}) classes.
Data of the type \texttt{LadokData} is not expected to change, unlike 
\texttt{LadokRemoteData}, which is.
Objects of type \texttt{LadokRemoteData} know how to update themselves, \ie fetch 
and refresh their data from LADOK.
When relevant they can also write data to LADOK, \ie update entries such as 
results.

One design criteria is to improve performance.
We do this by caching all factory methods of any \texttt{LadokSession}.
The \texttt{LadokSession} itself is also designed to be cacheable: if the session to 
LADOK expires, it will automatically reauthenticate to establish a new session.



\part{The library}

\input{../src/ladok3/ladok3.tex}


\part{API calls}

\input{../src/ladok3/api.tex}
\input{../src/ladok3/undoc.tex}



\part{A command-line interface}

\chapter{The base interface}

\input{../src/ladok3/cli.tex}

\chapter{The \texttt{data} command}

\input{../src/ladok3/data.tex}

\chapter{The \texttt{report} command}

\input{../src/ladok3/report.tex}

\chapter{The \texttt{student} command}

\input{../src/ladok3/student.tex}



\part{Other example applications}

\chapter{Transfer results from KTH Canvas to LADOK}

Here we provide an example program~\texttt{canvas2ladok.py} which exports 
results from KTH Canvas to LADOK for the introductory programming course~prgi 
(DD1315).
You can find an up-to-date version of this chapter at
\begin{center}
  \url{https://github.com/dbosk/intropy/tree/master/adm/reporting}.
\end{center}

\input{../examples/canvas2ladok.tex}


\backmatter
\printbibliography


\end{document}



\part{API calls}

\input{../src/ladok3/api.tex}
\input{../src/ladok3/undoc.tex}



\part{A command-line interface}

\chapter{The base interface}

\input{../src/ladok3/cli.tex}

\chapter{The \texttt{data} command}

\section{Data}\label{sec:data}

\subsection{Current Population Survey}

The Census Bureau administers the Current Population Survey Annual Social and Economic Supplement (CPS ASEC, or hereafter the CPS) each March. In March 2024, they surveyed 89,473 households representing the U.S. civilian non-institutional population about their activities in the 2023 calendar year.

The CPS's key strengths include:
\begin{itemize}
    \item Rich demographic detail including age, sex, race, ethnicity, and education
    \item Complete household relationship matrices
    \item Program participation indicators
    \item State identifiers, and partial county identifiers
\end{itemize}

However, the CPS has known limitations for tax modeling:
\begin{itemize}
    \item Underreporting of income, particularly at the top of the distribution due to top-coding
    \item Limited tax-relevant information (e.g., itemized deductions)
    \item No direct observation of tax units within households
    \item Imprecise measurement of certain income types (e.g., capital gains)
\end{itemize}

\subsection{IRS Public Use File}

The Internal Revenue Service Public Use File (PUF) is a national sample of individual income tax returns, representing the 151.2 million Form 1040, Form 1040A, and Form 1040EZ Federal Individual Income Tax Returns filed for Tax Year 2015. The file contains 119,675 records sampled at varying rates across strata, with 0.07 percent sampling for strata 7 through 13 \citep{bryant2023b}. The data are extensively transformed to protect taxpayer privacy while preserving statistical properties.

The Public Use Tax Demographic File supplements the PUF with:
\begin{itemize}
    \item Age ranges for primary taxpayers (different ranges for dependent vs non-dependent filers)
    \item Dependent age information in six categories (under 5, 5-13, 13-17, 17-19, 19-24, 24+)
    \item Gender of primary taxpayer
    \item Earnings splits for joint filers (categorizing primary earner share)
\end{itemize}

Key disclosure protections include:
\begin{itemize}
    \item Demographic information limited to returns in strata 7-13
    \item Suppression of dependent ages for returns with farm income or homebuyer credits
    \item Minimum population thresholds for dependent age reporting
    \item Sequential limits on dependent counts by filing status
\end{itemize}

The PUF's key strengths include:
\begin{itemize}
    \item Precise income amounts derived from information returns
    \item Complete tax return information including itemized deductions
    \item Actual tax unit structure
    \item Accurate income type classification
\end{itemize}

The PUF's limitations include:
\begin{itemize}
    \item Limited demographic information
    \item No household structure beyond the tax unit
    \item No geographic information such as state
    \item No program participation information
    \item Privacy protections that mask extreme values
    \item Lag; the latest version as of November 2024 is for the 2015 tax year
\end{itemize}

\subsection{External Validation Sources}

We validate our enhanced dataset against 570 targets from several external sources:

\subsubsection{IRS Statistics of Income}

The Statistics of Income (SOI) Division publishes detailed tabulations of tax return data, including:
\begin{itemize}
    \item Income amounts by source and adjusted gross income bracket
    \item Number of returns by filing status
    \item Itemized deduction amounts and counts
    \item Tax credits and their distribution
\end{itemize}

These tabulations serve as key targets in our reweighting procedure and validation metrics.

\subsubsection{CPS ASEC Public Tables}

Census Bureau publications provide demographic and program participation benchmarks, including:
\begin{itemize}
    \item Age distribution by state
    \item Household size distribution
    \item Program participation rates
\end{itemize}

\subsubsection{Administrative Program Totals}

We incorporate official totals from various agencies, including but not limited to:
\begin{itemize}
    \item Social Security Administration beneficiary counts and benefit amounts
    \item SNAP participation and benefits from USDA
    \item Earned Income Tax Credit statistics from IRS
    \item Unemployment Insurance claims and benefits from Department of Labor
\end{itemize}

\subsection{Variable Harmonization}

A crucial preparatory step is harmonizing variables across datasets. We develop a detailed crosswalk between CPS and PUF variables, accounting for definitional differences. Key considerations include:
\begin{itemize}
    \item Income classification (e.g., business vs. wage income)
    \item Geographic definitions
    \item Family relationship categories
\end{itemize}

For some variables, direct correspondence is impossible, requiring imputation strategies described in the methodology section. The complete variable crosswalk is available in our open-source repository.

\chapter{The \texttt{report} command}

\input{../src/ladok3/report.tex}

\chapter{The \texttt{student} command}

\input{../src/ladok3/student.tex}



\part{Other example applications}

\chapter{Transfer results from KTH Canvas to LADOK}

Here we provide an example program~\texttt{canvas2ladok.py} which exports 
results from KTH Canvas to LADOK for the introductory programming course~prgi 
(DD1315).
You can find an up-to-date version of this chapter at
\begin{center}
  \url{https://github.com/dbosk/intropy/tree/master/adm/reporting}.
\end{center}

\input{../examples/canvas2ladok.tex}


\backmatter
\printbibliography


\end{document}



\part{API calls}

\input{../src/ladok3/api.tex}
\input{../src/ladok3/undoc.tex}



\part{A command-line interface}

\chapter{The base interface}

\input{../src/ladok3/cli.tex}

\chapter{The \texttt{data} command}

\section{Data}\label{sec:data}

\subsection{Current Population Survey}

The Census Bureau administers the Current Population Survey Annual Social and Economic Supplement (CPS ASEC, or hereafter the CPS) each March. In March 2024, they surveyed 89,473 households representing the U.S. civilian non-institutional population about their activities in the 2023 calendar year.

The CPS's key strengths include:
\begin{itemize}
    \item Rich demographic detail including age, sex, race, ethnicity, and education
    \item Complete household relationship matrices
    \item Program participation indicators
    \item State identifiers, and partial county identifiers
\end{itemize}

However, the CPS has known limitations for tax modeling:
\begin{itemize}
    \item Underreporting of income, particularly at the top of the distribution due to top-coding
    \item Limited tax-relevant information (e.g., itemized deductions)
    \item No direct observation of tax units within households
    \item Imprecise measurement of certain income types (e.g., capital gains)
\end{itemize}

\subsection{IRS Public Use File}

The Internal Revenue Service Public Use File (PUF) is a national sample of individual income tax returns, representing the 151.2 million Form 1040, Form 1040A, and Form 1040EZ Federal Individual Income Tax Returns filed for Tax Year 2015. The file contains 119,675 records sampled at varying rates across strata, with 0.07 percent sampling for strata 7 through 13 \citep{bryant2023b}. The data are extensively transformed to protect taxpayer privacy while preserving statistical properties.

The Public Use Tax Demographic File supplements the PUF with:
\begin{itemize}
    \item Age ranges for primary taxpayers (different ranges for dependent vs non-dependent filers)
    \item Dependent age information in six categories (under 5, 5-13, 13-17, 17-19, 19-24, 24+)
    \item Gender of primary taxpayer
    \item Earnings splits for joint filers (categorizing primary earner share)
\end{itemize}

Key disclosure protections include:
\begin{itemize}
    \item Demographic information limited to returns in strata 7-13
    \item Suppression of dependent ages for returns with farm income or homebuyer credits
    \item Minimum population thresholds for dependent age reporting
    \item Sequential limits on dependent counts by filing status
\end{itemize}

The PUF's key strengths include:
\begin{itemize}
    \item Precise income amounts derived from information returns
    \item Complete tax return information including itemized deductions
    \item Actual tax unit structure
    \item Accurate income type classification
\end{itemize}

The PUF's limitations include:
\begin{itemize}
    \item Limited demographic information
    \item No household structure beyond the tax unit
    \item No geographic information such as state
    \item No program participation information
    \item Privacy protections that mask extreme values
    \item Lag; the latest version as of November 2024 is for the 2015 tax year
\end{itemize}

\subsection{External Validation Sources}

We validate our enhanced dataset against 570 targets from several external sources:

\subsubsection{IRS Statistics of Income}

The Statistics of Income (SOI) Division publishes detailed tabulations of tax return data, including:
\begin{itemize}
    \item Income amounts by source and adjusted gross income bracket
    \item Number of returns by filing status
    \item Itemized deduction amounts and counts
    \item Tax credits and their distribution
\end{itemize}

These tabulations serve as key targets in our reweighting procedure and validation metrics.

\subsubsection{CPS ASEC Public Tables}

Census Bureau publications provide demographic and program participation benchmarks, including:
\begin{itemize}
    \item Age distribution by state
    \item Household size distribution
    \item Program participation rates
\end{itemize}

\subsubsection{Administrative Program Totals}

We incorporate official totals from various agencies, including but not limited to:
\begin{itemize}
    \item Social Security Administration beneficiary counts and benefit amounts
    \item SNAP participation and benefits from USDA
    \item Earned Income Tax Credit statistics from IRS
    \item Unemployment Insurance claims and benefits from Department of Labor
\end{itemize}

\subsection{Variable Harmonization}

A crucial preparatory step is harmonizing variables across datasets. We develop a detailed crosswalk between CPS and PUF variables, accounting for definitional differences. Key considerations include:
\begin{itemize}
    \item Income classification (e.g., business vs. wage income)
    \item Geographic definitions
    \item Family relationship categories
\end{itemize}

For some variables, direct correspondence is impossible, requiring imputation strategies described in the methodology section. The complete variable crosswalk is available in our open-source repository.

\chapter{The \texttt{report} command}

\input{../src/ladok3/report.tex}

\chapter{The \texttt{student} command}

\input{../src/ladok3/student.tex}



\part{Other example applications}

\chapter{Transfer results from KTH Canvas to LADOK}

Here we provide an example program~\texttt{canvas2ladok.py} which exports 
results from KTH Canvas to LADOK for the introductory programming course~prgi 
(DD1315).
You can find an up-to-date version of this chapter at
\begin{center}
  \url{https://github.com/dbosk/intropy/tree/master/adm/reporting}.
\end{center}

\input{../examples/canvas2ladok.tex}


\backmatter
\printbibliography


\end{document}



\part{API calls}

\input{../src/ladok3/api.tex}
\input{../src/ladok3/undoc.tex}



\part{A command-line interface}

\chapter{The base interface}

\input{../src/ladok3/cli.tex}

\chapter{The \texttt{data} command}

\section{Data}\label{sec:data}

\subsection{Current Population Survey}

The Census Bureau administers the Current Population Survey Annual Social and Economic Supplement (CPS ASEC, or hereafter the CPS) each March. In March 2024, they surveyed 89,473 households representing the U.S. civilian non-institutional population about their activities in the 2023 calendar year.

The CPS's key strengths include:
\begin{itemize}
    \item Rich demographic detail including age, sex, race, ethnicity, and education
    \item Complete household relationship matrices
    \item Program participation indicators
    \item State identifiers, and partial county identifiers
\end{itemize}

However, the CPS has known limitations for tax modeling:
\begin{itemize}
    \item Underreporting of income, particularly at the top of the distribution due to top-coding
    \item Limited tax-relevant information (e.g., itemized deductions)
    \item No direct observation of tax units within households
    \item Imprecise measurement of certain income types (e.g., capital gains)
\end{itemize}

\subsection{IRS Public Use File}

The Internal Revenue Service Public Use File (PUF) is a national sample of individual income tax returns, representing the 151.2 million Form 1040, Form 1040A, and Form 1040EZ Federal Individual Income Tax Returns filed for Tax Year 2015. The file contains 119,675 records sampled at varying rates across strata, with 0.07 percent sampling for strata 7 through 13 \citep{bryant2023b}. The data are extensively transformed to protect taxpayer privacy while preserving statistical properties.

The Public Use Tax Demographic File supplements the PUF with:
\begin{itemize}
    \item Age ranges for primary taxpayers (different ranges for dependent vs non-dependent filers)
    \item Dependent age information in six categories (under 5, 5-13, 13-17, 17-19, 19-24, 24+)
    \item Gender of primary taxpayer
    \item Earnings splits for joint filers (categorizing primary earner share)
\end{itemize}

Key disclosure protections include:
\begin{itemize}
    \item Demographic information limited to returns in strata 7-13
    \item Suppression of dependent ages for returns with farm income or homebuyer credits
    \item Minimum population thresholds for dependent age reporting
    \item Sequential limits on dependent counts by filing status
\end{itemize}

The PUF's key strengths include:
\begin{itemize}
    \item Precise income amounts derived from information returns
    \item Complete tax return information including itemized deductions
    \item Actual tax unit structure
    \item Accurate income type classification
\end{itemize}

The PUF's limitations include:
\begin{itemize}
    \item Limited demographic information
    \item No household structure beyond the tax unit
    \item No geographic information such as state
    \item No program participation information
    \item Privacy protections that mask extreme values
    \item Lag; the latest version as of November 2024 is for the 2015 tax year
\end{itemize}

\subsection{External Validation Sources}

We validate our enhanced dataset against 570 targets from several external sources:

\subsubsection{IRS Statistics of Income}

The Statistics of Income (SOI) Division publishes detailed tabulations of tax return data, including:
\begin{itemize}
    \item Income amounts by source and adjusted gross income bracket
    \item Number of returns by filing status
    \item Itemized deduction amounts and counts
    \item Tax credits and their distribution
\end{itemize}

These tabulations serve as key targets in our reweighting procedure and validation metrics.

\subsubsection{CPS ASEC Public Tables}

Census Bureau publications provide demographic and program participation benchmarks, including:
\begin{itemize}
    \item Age distribution by state
    \item Household size distribution
    \item Program participation rates
\end{itemize}

\subsubsection{Administrative Program Totals}

We incorporate official totals from various agencies, including but not limited to:
\begin{itemize}
    \item Social Security Administration beneficiary counts and benefit amounts
    \item SNAP participation and benefits from USDA
    \item Earned Income Tax Credit statistics from IRS
    \item Unemployment Insurance claims and benefits from Department of Labor
\end{itemize}

\subsection{Variable Harmonization}

A crucial preparatory step is harmonizing variables across datasets. We develop a detailed crosswalk between CPS and PUF variables, accounting for definitional differences. Key considerations include:
\begin{itemize}
    \item Income classification (e.g., business vs. wage income)
    \item Geographic definitions
    \item Family relationship categories
\end{itemize}

For some variables, direct correspondence is impossible, requiring imputation strategies described in the methodology section. The complete variable crosswalk is available in our open-source repository.

\chapter{The \texttt{report} command}

\input{../src/ladok3/report.tex}

\chapter{The \texttt{student} command}

\input{../src/ladok3/student.tex}



\part{Other example applications}

\chapter{Transfer results from KTH Canvas to LADOK}

Here we provide an example program~\texttt{canvas2ladok.py} which exports 
results from KTH Canvas to LADOK for the introductory programming course~prgi 
(DD1315).
You can find an up-to-date version of this chapter at
\begin{center}
  \url{https://github.com/dbosk/intropy/tree/master/adm/reporting}.
\end{center}

\input{../examples/canvas2ladok.tex}


\backmatter
\printbibliography


\end{document}
